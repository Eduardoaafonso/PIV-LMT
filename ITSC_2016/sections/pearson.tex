
\subsection{THE PEARSON CORRELATION COEFFICIENT}

$PCC$ is used in statistical analyses, pattern recognition and computer vision. 
It can be used to comparing two images in an object recognition system. 
The following equation describes the calculus of $PCC$ between two gray scale digital images\cite{Eugene},
represented by the matrices $A$ and $B$ with $M$ elements each one,
\begin{equation}
r = \frac{\sum \limits_{i}^{M} (a_i-\mu_a)(b_i-\mu_b)}{\sqrt{\sum \limits_{i}^{M} (a_i-\mu_a)^2} \sqrt{\sum\limits_{i}^{M} (b_i-\mu_b)^2}}.
\end{equation}

Where, $a_i$ is the intensity of the i-th pixel in the  matrix $A$, 
$b_i$ is the intensity of the i-th pixel in the matrix $B$, 
$\mu_a$ is the mean intensity of $A$,
$\mu_b$ is the mean intensity of $B$ and
$r$ is the correlation coefficient\cite{Miranda Neto}.
