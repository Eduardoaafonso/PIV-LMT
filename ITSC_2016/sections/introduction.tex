\section{INTRODUCTION}

Traffic accident has been an important point to government. The world average of people
died at traffic is around $10$ people per $100$ thousand, but in some countries, it can be worse.
In 2016, the group which compose the United Nations General Assembly adopted a resolution for
improve global road safety and they consider that $2011$-$2020$ is decade of action for road safety.
Traffic jam is other problem faced by cities, it is caused by reckless drivers and accidents.
Many centers of research have been studying solutions to improve this condition and decrease the 
numbers of death and accidents.

Monocular vision is an emerging field in autonomous vehicles. In this sense,
several application have presented solutions to current problems, 
however systems  of low computational cost remain a challenge.

Field Programmable Gate Arrays ($FPGA$) has been used in several application in computer vision. The article \cite{Honegger} 
is an example of these kind studies. \cite{Honegger} presents a platform process date of 
cameras configured in $127$ frames per second and resolution $376$x$240$. It calculates radial distortions, disparity estimation
and streaming of date. Embedded systems adopt $FPGA$ for low weight. Thus, there are some advances in this research 
that prove efficiency of this platform. System implemented is capable of calculate a relative velocity of any object 
in the scene. It uses a relative distance between camera and object to estimate velocity on optic axis ($Z$) and axis $X$ and $Y$.

Other approach about monocular distance and velocity estimation is showed in \cite{Breugel}. Authors propose a conception of 
observability in image, it would be basic requirement to estimate velocity and distance. Inputs not null are identified by
equation observability Gramian to linear systems and Lie Algebraic tools to non-linear systems. If 
number of terms linearly independents is equal of number of state system, so system is classified like observability and, 
consequently, velocity and distance must not be zero.

These studies contribute in field of vision monocular to estimation of distance and velocity of object in scene. 
Our contribution is an algorithm that calculate the relative distance using Particle Image Velocimetry ($PIV$).


The $PIV$ \cite{Bastiaans} technique is used in many fields of 
knowledge, \cite{Story, Xu}, to calculate the velocity of fluids in different parts. 
Here, $PIV$ was adjusted for the case of autonomous vehicles using matching criteria based on 
the Pearson Correlation Coefficient ($PCC$)\cite{Miranda Neto} over the POV-Ray \cite{povray} program.

%We define some variables used in the paper. The object of interesting, or target, is called  
%of Region of Interesting ($ROI$), 
%the Window of Search ($WOS$) is a rectangle where the program will search the $ROI$. 

