\begin{comment}

\section{INTRODUCTION}

Traffic accident has been an important point to government. The world average of people
died at traffic is around $10$ people per $100$ thousand, but in some countries, it can be worse.
In 2016, the group which compose the United Nations General Assembly adopted a resolution for
improve global road safety and they consider that $2011$-$2020$ is the decade of action for road safety.
Traffic jam is other problem faced by cities, it is caused by reckless drivers and accidents.
Many centers of research have been studying solutions to improve this condition and decrease the 
numbers of death and accidents.

Monocular vision is an emerging field in autonomous vehicles. In this sense,
several application have presented solutions to current problems, 
however systems  of low computational cost remain a challenge.

Field Programmable Gate Arrays ($FPGA$) has been used in several application in computer vision. The article \cite{Honegger} 
show a research that use it. \cite{Honegger} presents a platform process date of 
cameras configured in $127$ frames per second and resolution $376$x$240$. It calculates radial distortions, disparity estimation
and streaming of date. Embedded systems adopt $FPGA$ for low weight. Thus, there are some advances in this research 
that prove efficiency of this platform. System implemented is capable to calculate a relative velocity of any object 
in the scene. It uses a relative distance between camera and object to estimate velocity on optic axis ($Z$) and axis $X$ and $Y$.

Other approach about monocular distance and velocity estimation is showed in \cite{Breugel}. Authors propose a conception of 
observability in image, it would be basic requirement to estimate velocity and distance. Inputs not null are identified by
equation observability Gramian to linear systems and Lie Algebraic tools to non-linear systems. If 
number of terms linearly independents is equal of number of state system, so system is classified like observability and, 
consequently, velocity and distance must not be zero.

These studies contribute in field of vision monocular to estimate the distance and velocity of object in scene. 
Our contribution is an algorithm that calculate the relative distance using Particle Image Velocimetry ($PIV$).


The $PIV$ \cite{Bastiaans} technique is used in many fields of 
knowledge, \cite{Story, Xu}, to calculate the velocity of fluids in different parts. 
Here, $PIV$ was adjusted for the case of autonomous vehicles using matching criteria based on 
the Pearson Correlation Coefficient ($PCC$)\cite{Miranda Neto} over the POV-Ray \cite{povray} program.

\end{comment}
%%%%%%%%%%%%%%%%%%%%%%%%%%%%%%%%%%%%%%%%%%%portugues%%%%%%%%%%%%%%%%%%%%%%%%%%%%%%%%%%%%%%%%%%%%%%%%%%%%%%%%%%%%%%%%

\section{INTRODUCTION}

Os acidentes no trânsito têm se tornado um problema de governo, pois há uma quantidade significativa de pessoas que
perdem suas vidas neste tipo de acidente. Em 2016, a United Nations General Assembly adotou uma resolução que incorporam medidas que visam 
a melhoria da segurança em trânsitos na esfera global. O período de 2011 à 2020 está sendo considerado a década da ação a favor
de um trânsito mais segurança.

Os congestionamentos é um dos agravadores do risco de morte e acidentes em trânsito, além de falhas sejam humanas ou não. 
Face à essas questões, vários grupos de pesquisas debruçam seus estudos para encontrar soluções viáveis e, desta maneira,
reduzir o número de acidentes e mortes no trânsito com a contribuição de diversos aparatos tecnológicos, entre eles: as câmeras.

A visão monocular é uma área que vem solucionando diversas questões no âmbito de veículos autônomos e, por conseguinte, 
contribui para a criação de sistemas mais seguros. Entretanto, o baixo custo computacional ainda é um desavio 
para as atuais tecnologias. O estado da arte mostra o esforço dos grupos de pesquisas afim de que reduza o custo computacional e,
também, energético dos sistemas associando critérios para as tomadas de decisões.

Field Programmable Gate Arrays (FPGA) é um dispositivo lógico programável que vem sendo muito utilizado em análise de imagens. \cite{Honegger}
apresenta uma plataforma para processamento de dados de câmeras configuradas em $127$ frames por segundo com resolução de $376x240$. Essa 
implementação permite calcular a distorção radial do objeto analisado, estimação da disparidade e o fluxo de dados. Além da 
velocidade relativa dos eixos ópticos (X, Y e Z).

O conceito de "observabilidade da imagem" é proposto em \cite{Breugel}, o qual sugere requisitos básicos para a estimativa de velocidade e 
distância de um objeto. Nesta implementação, as entradas nulas são definidas pela equação de sistemas lineares de observabilidade Gramian e,
para sistemas não-lineares, usa-se as ferramentas de Lie Algebraic. A condição que permite vincular a observabilidade do sistema é dado por:
se o número de termos linearmente indepentes é igual ao número de estados, então o sistema é observável e, por consequência, a velocidade e a 
distância não devem ser zeros.

Os artigos mostrados como estado da arte permite observar que há uma pesquisa de alto nível sobre o assunto de visão computacional associada à
estimativa de velocidade e distância de objetos. A contribuição deste artigo para a área é o uso da técnica Particle Image Velocimetry (PIV) 
para a estimativa da velocidade e distância relativas de objetos na cena.

PIV \cite{Bastiaans} é uma técnica originariamente usada para determinar velocidade de fluídos, contudo ela foi ajustada para a aplicação em
veículos autônomos usando critério de busca baseados no coeficiente de correlação de Pearson (PCC em inglês) \cite{Miranda Neto}, simulado 
no POV-Ray \cite{povray}.


















