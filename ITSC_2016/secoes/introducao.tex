\begin{comment}

\section{INTRODUÇÃO}

Traffic accident has been an important point to government. The world average of people
died at traffic is around $10$ people per $100$ thousand, but in some countries, it can be worse.
In 2016, the group which compose the United Nations General Assembly adopted a resolution for
improve global road safety and they consider that $2011$-$2020$ is the decade of action for road safety.
Traffic jam is other problem faced by cities, it is caused by reckless drivers and accidents.
Many centers of research have been studying solutions to improve this condition and decrease the 
numbers of death and accidents.

Monocular vision is an emerging field in autonomous vehicles. In this sense,
several application have presented solutions to current problems, 
however systems  of low computational cost remain a challenge.

Field Programmable Gate Arrays ($FPGA$) has been used in several application in computer vision. The article \cite{Honegger} 
show a research that use it. \cite{Honegger} presents a platform process date of 
cameras configured in $127$ frames per second and resolution $376$x$240$. It calculates radial distortions, disparity estimation
and streaming of date. Embedded systems adopt $FPGA$ for low weight. Thus, there are some advances in this research 
that prove efficiency of this platform. System implemented is capable to calculate a relative velocity of any object 
in the scene. It uses a relative distance between camera and object to estimate velocity on optic axis ($Z$) and axis $X$ and $Y$.

Other approach about monocular distance and velocity estimation is showed in \cite{Breugel}. Authors propose a conception of 
observability in image, it would be basic requirement to estimate velocity and distance. Inputs not null are identified by
equation observability Gramian to linear systems and Lie Algebraic tools to non-linear systems. If 
number of terms linearly independents is equal of number of state system, so system is classified like observability and, 
consequently, velocity and distance must not be zero.

These studies contribute in field of vision monocular to estimate the distance and velocity of object in scene. 
Our contribution is an algorithm that calculate the relative distance using Particle Image Velocimetry ($PIV$).


The $PIV$ \cite{Bastiaans} technique is used in many fields of 
knowledge, \cite{Story, Xu}, to calculate the velocity of fluids in different parts. 
Here, $PIV$ was adjusted for the case of autonomous vehicles using matching criteria based on 
the Pearson Correlation Coefficient ($CCP$)\cite{Miranda Neto} over the POV-Ray \cite{povray} program.

\end{comment}
%%%%%%%%%%%%%%%%%%%%%%%%%%%%%%%%%%%%%%%%%%%portugues%%%%%%%%%%%%%%%%%%%%%%%%%%%%%%%%%%%%%%%%%%%%%%%%%%%%%%%%%%%%%%%%

\section{INTRODUÇÃO}

Acidentes de trânsito têm se tornado um problema de estado, pois há uma quantidade significativa de pessoas que
perdem suas vidas neste tipo de acidente. Em 2016, a Organização das Nações Unidas adotou uma 
resolução que incorpora medidas que visam 
a melhoria da segurança no trânsito com um programa global; 
o período de 2011 a 2020 está sendo considerado a década da ação a favor
de um trânsito com maior segurança.
Face à essas questões, vários grupos de pesquisas debruçam seus estudos 
para encontrar soluções viáveis e, desta maneira
reduzir o número de mortes em acidentes de trânsito, com a 
utilização de diversos aparatos tecnológicos, entre eles: as câmeras inteligentes.


Diversos sensores podem ser usados no contexto de veículos autônomos. Dentre eles, 
câmeras são considerados sensores de baixo-custo e são amplamente utilizadas. 
Estas, integradas com outros sensores, podem produzir resultados relevantes e complementares devido à 
quantidade de informações disponíveis nas imagens, como cores, texturas, etc. 
Se por um lado, os sistemas baseados em visão monocular não dependem de processos 
complexos de calibração, obter a informação relativa à profundidade pode ser uma difícil 
tarefa.


Neste sentido, o conceito de ``observabilidade da imagem'' é 
proposto em \cite{Breugel}, onde são apresentados requisitos 
básicos para a estimativa de velocidade e 
distância de um objeto. Nesta implementação, as entradas nulas 
são definidas por uma equação de sistemas lineares de observabilidade Gramian e,
para sistemas não-lineares, usa-se as ferramentas de Lie Algebraic. 
A condição que permite vincular a observabilidade do sistema é dada por:
se o número de termos linearmente independentes é igual ao número de estados, 
então o sistema é observável e, por consequência, a velocidade e a 
distância não devem ser zeros.

A solução para a estimativa de velocidade e distância de objetos em uma imagem é 
apresentada neste artigo à partir do uso da técnica 
conhecida como velocimetria por imagem de partículas ou PIV 
(do inglês Particle Image Velocimetry). 
O $PIV$ é originalmente usado para determinar a velocidade de fluídos; contudo, neste trabalho
a técnica foi ajustada para aplicação em visão monocular. Para isso, o Coeficiente de 
Correlação de Pearson (CCP) \cite{Miranda Neto} foi usado. Simulações 
para a prova do algoritmo proposto foram realizadas utilizando POV-Ray \cite{povray}.




















