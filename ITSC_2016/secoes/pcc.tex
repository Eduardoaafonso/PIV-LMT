\begin{comment}

\subsection{THE PEARSON CORRELATION COEFFICIENT}

$PCC$ is used in statistical analyses, pattern recognition and computer vision. 
It can be used to comparing two images in an object recognition system. 
The following equation describes the $PCC$ method for two gray scale digital images\cite{Eugene},
represented by the matrices $A$ and $B$ with $M$ elements each one,
\begin{equation}
r = \frac{\sum \limits_{i}^{M} (a_i-\mu_a)(b_i-\mu_b)}{\sqrt{\sum \limits_{i}^{M} (a_i-\mu_a)^2} \sqrt{\sum\limits_{i}^{M} (b_i-\mu_b)^2}},
\end{equation}
\begin{equation}\label{eq:PCC}
 r=PCC(A,B)
\end{equation}

where $a_i$ is the intensity of the i-th pixel in the  matrix $A$, 
$b_i$ is the intensity of the i-th pixel in the matrix $B$, 
$\mu_a$ is the mean intensity of $A$,
$\mu_b$ is the mean intensity of $B$ and
$r$ is the correlation coefficient \cite{Miranda Neto}.

\end{comment}

%%%%%%%%%%%%%%%%%%%%%%%%%%%%%%%%%%%%portugues%%%%%%%%%%%%%%%%%%%%%%%%%%%%%%%%%%%%%%%%


\subsection{Coeficiente de correla��o de Pearson - PCC}

PCC � uma ferramenta de compara��o sendo nesse caso usada em imagens. H� v�rias aplica��es desse m�todo 
em diversas �reas, tais como: an�lise estat�stica, reconhecimento de padr�o e vis�o computacional. 
A equa��o seguinte descreve como � feita essa compara��o, sendo espec�ficado para imagens digitais 
em tons de cinzas \cite{Eugene}, representado pelas matrizes $A$ e $B$ com $M$ elementos.

\begin{equation}
r = \frac{\sum \limits_{i}^{M} (a_i-\mu_a)(b_i-\mu_b)}{\sqrt{\sum \limits_{i}^{M} (a_i-\mu_a)^2} \sqrt{\sum\limits_{i}^{M} (b_i-\mu_b)^2}},
\end{equation}
\begin{equation}\label{eq:PCC}
 r=PCC(A,B)
\end{equation}

Sendo $a_i$ a intensidade do i-�simo pixel na matriz $A$, $b_i$ a intensidade do i-�simo pixel na matriz
$B$, $\mu_a$ a m�dia das intensidades de $A$ e $\mu_b$ a m�dia das intensidades de $B$ e $r$ � o coeficiente
de correla��o \cite{Miranda Neto}.