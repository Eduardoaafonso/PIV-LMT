\begin{comment}

\subsection{THE PEARSON CORRELATION COEFFICIENT}

$PCC$ is used in statistical analyses, pattern recognition and computer vision. 
It can be used to comparing two images in an object recognition system. 
The following equation describes the $PCC$ method for two gray scale digital images\cite{Eugene},
represented by the matrices $A$ and $B$ with $M$ elements each one,
\begin{equation}
r = \frac{\sum \limits_{i}^{M} (a_i-\mu_a)(b_i-\mu_b)}{\sqrt{\sum \limits_{i}^{M} (a_i-\mu_a)^2} \sqrt{\sum\limits_{i}^{M} (b_i-\mu_b)^2}},
\end{equation}
\begin{equation}\label{eq:PCC}
 r=PCC(A,B)
\end{equation}

where $a_i$ is the intensity of the i-th pixel in the  matrix $A$, 
$b_i$ is the intensity of the i-th pixel in the matrix $B$, 
$\mu_a$ is the mean intensity of $A$,
$\mu_b$ is the mean intensity of $B$ and
$r$ is the correlation coefficient \cite{Miranda Neto}.

\end{comment}

%%%%%%%%%%%%%%%%%%%%%%%%%%%%%%%%%%%%portugues%%%%%%%%%%%%%%%%%%%%%%%%%%%%%%%%%%%%%%%%


\subsection{Coeficiente de correlação de Pearson - PCC}

PCC é uma ferramenta de comparação sendo nesse caso usada em imagens. Há várias aplicações desse método 
em diversas áreas, tais como: análise estatística, reconhecimento de padrões e visão computacional. 
A equação seguinte descreve como é feita essa comparação, para duas imagens digitais 
em tons de cinzas \cite{Eugene}, representadas pelas matrizes $A$ e $B$ com $M$ elementos cada um.

\begin{equation}
r = \frac{\sum \limits_{i}^{M} (a_i-\mu_a)(b_i-\mu_b)}{\sqrt{\sum \limits_{i}^{M} (a_i-\mu_a)^2} \sqrt{\sum\limits_{i}^{M} (b_i-\mu_b)^2}},
\end{equation}
\begin{equation}\label{eq:PCC}
 r=PCC(A,B)
\end{equation}

Sendo $a_i$ a intensidade do i-ésimo pixel na matriz $A$, $b_i$ a intensidade do i-ésimo pixel na matriz
$B$, $\mu_a$ a média das intensidades de $A$, $\mu_b$ a média das intensidades de $B$ e $r$ é o coeficiente
de correlação \cite{Miranda Neto}.
