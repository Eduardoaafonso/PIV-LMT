
\begin{comment}
From the presented examples,
it can be observed that one application that uses the tracking
and the departure factor is related with the risk of collision.
It is possible to estimate how of fast an object is departing.
Thus, if the  departure factor tends to zero or 
if the velocity of departure factor changes to lower negatives values every time, 
probably, there is a high risk of collision. The $PIV$ technique has presented satisfactory results. 
It can be concluded that estimating collision using velocity of departure factor, 
tracking of objects in 2 or 3 dimensions, and departure distance
relative to the first position of $ROI$. 
The simulations in both cases has given promissory results.
\end{comment}

Dos exemplos apresentados, pode-se observar que uma implementação 
que obtenha uma rota em $3D$ de um objeto e o 
fator de aproximação, como o algoritmo proposto, nos proporciona um método para avaliar o risco de colisão.
Assim, é possível estimar o rápido afastamento de um objeto de interesse;
de modo que, se o fator de aproximação tende a zero ou se a velocidade de cambio 
do fator de aproximação muda 
para valores negativos mais baixos de cada vez; isto indica que
provavelmente existe um alto risco de colisão.
A técnica $PIV$ tem mostrando resultados satisfatórios no que concerne estimação 
de rotas de objetos em colisão ao observador.
