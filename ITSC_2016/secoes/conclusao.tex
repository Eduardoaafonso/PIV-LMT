
\begin{comment}
From the presented examples,
it can be observed that one application that uses the tracking
and the departure factor is related with the risk of collision.
It is possible to estimate how of fast an object is departing.
Thus, if the  departure factor tends to zero or 
if the velocity of departure factor changes to lower negatives values every time, 
probably, there is a high risk of collision. The $PIV$ technique has presented satisfactory results. 
It can be concluded that estimating collision using velocity of departure factor, 
tracking of objects in 2 or 3 dimensions, and departure distance
relative to the first position of $ROI$. 
The simulations in both cases has given promissory results.
\end{comment}

Dos testes apresentados neste artigo, pode-se observar que o algoritmo
 proporciona uma descrição, relativa ao observador, 
da trajetória do objeto de interesse em 3 dimensões.
Por outro lado, mediante a taxa de mudança do fator de aproximação, é possível  obter 
um critério para avaliar o risco de colisão contra o observador;
de modo que, se o fator de aproximação tende a zero, ou se a velocidade de mudança
do fator de aproximação leva 
para valores negativos, isto indica que
existe um alto risco de colisão.
Assim a técnica $PIV$ aplicada a visão monocular apresenta resultados satisfatórios 
na estimativa de trajetórias e do risco de colisão dos objetos de interesse.

