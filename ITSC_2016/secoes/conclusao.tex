
\begin{comment}
From the presented examples,
it can be observed that one application that uses the tracking
and the departure factor is related with the risk of collision.
It is possible to estimate how of fast an object is departing.
Thus, if the  departure factor tends to zero or 
if the velocity of departure factor changes to lower negatives values every time, 
probably, there is a high risk of collision. The $PIV$ technique has presented satisfactory results. 
It can be concluded that estimating collision using velocity of departure factor, 
tracking of objects in 2 or 3 dimensions, and departure distance
relative to the first position of $ROI$. 
The simulations in both cases has given promissory results.
\end{comment}

Dos exemplos apresentados, pode-se observar que um aplicativo 
que usa o rastreamento e o fator de aproximação está relacionado com o risco de colisão.
Assim, é possível estimar o rápido afastamento de um objeto;
de modo que, se o fator de partida tende a zero ou se a velocidade do fator de partida muda 
para valores negativos mais baixos de cada vez,
provavelmente, existe um alto risco de colisão.
A técnica $PIV$ tem mostrando resultados satisfatórios no que concerne estimação de colisão usando 
fator de aproximação e velocidade relativa em 2 e 3 dimensões. As simulações feitas demonstram 
resultados promissores no estudo desenvolvido.
