
\begin{comment}
From the presented examples,
it can be observed that one application that uses the tracking
and the departure factor is related with the risk of collision.
It is possible to estimate how of fast an object is departing.
Thus, if the  departure factor tends to zero or 
if the velocity of departure factor changes to lower negatives values every time, 
probably, there is a high risk of collision. The $PIV$ technique has presented satisfactory results. 
It can be concluded that estimating collision using velocity of departure factor, 
tracking of objects in 2 or 3 dimensions, and departure distance
relative to the first position of $ROI$. 
The simulations in both cases has given promissory results.
\end{comment}

O risco de colisão é relacionado com o $tracking$ e o fator de aproximação do objeto nos 
exemplos apresentados. Deste modo, é possível estimar o quanto rápido o objeto se aproxima, pois
se o fator se aproxima de zero ou é menor que zero frequentemente, então a probabilidade de colisão é alto. 
A técnica $PIV$ tem monstrado resultados satisfatórios no que concerne estimação de colisão usando 
fator de aproximação e velocidade relativa em 2 e 3 dimensões. As simulações feitas demonstram 
resultados promissores no estudo desenvolvido.