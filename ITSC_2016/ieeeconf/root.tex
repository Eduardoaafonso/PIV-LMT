%%%%%%%%%%%%%%%%%%%%%%%%%%%%%%%%%%%%%%%%%%%%%%%%%%%%%%%%%%%%%%%%%%%%%%%%%%%%%%%%
%2345678901234567890123456789012345678901234567890123456789012345678901234567890
%        1         2         3         4         5         6         7         8

\documentclass[letterpaper, 10 pt, conference]{ieeeconf}  % Comment this line out if you need a4paper

%\documentclass[a4paper, 10pt, conference]{ieeeconf}      % Use this line for a4 paper

\IEEEoverridecommandlockouts                              % This command is only needed if 
                                                          % you want to use the \thanks command

\overrideIEEEmargins                                      % Needed to meet printer requirements.

% See the \addtolength command later in the file to balance the column lengths
% on the last page of the document

% The following packages can be found on http:\\www.ctan.org
%\usepackage{graphics} % for pdf, bitmapped graphics files
%\usepackage{epsfig} % for postscript graphics files
%\usepackage{mathptmx} % assumes new font selection scheme installed
%\usepackage{times} % assumes new font selection scheme installed
%\usepackage{amsmath} % assumes amsmath package installed
%\usepackage{amssymb}  % assumes amsmath package installed

\title{\LARGE \bf
Preparation of Papers for IEEE Sponsored Conferences \& Symposia*
}


\author{Author$^{1}$ and Researcher$^{2}$% <-this % stops a space
\thanks{*This work was not supported by any organization}% <-this % stops a space
\thanks{$^{1}$ Author is with Faculty of Electrical Engineering, Mathematics and Computer Science,
        University of Twente, 7500 AE Enschede, The Netherlands
        {\tt\small albert.author@papercept.net}}%
\thanks{$^{2}$ Researcheris with the Department of Electrical Engineering, Wright State University,
        Dayton, OH 45435, USA
        {\tt\small b.d.researcher@ieee.org}}%
}


\begin{document}



\maketitle
\thispagestyle{empty}
\pagestyle{empty}


%%%%%%%%%%%%%%%%%%%%%%%%%%%%%%%%%%%%%%%%%%%%%%%%%%%%%%%%%%%%%%%%%%%%%%%%%%%%%%%%
\begin{abstract}


\end{abstract}


%%%%%%%%%%%%%%%%%%%%%%%%%%%%%%%%%%%%%%%%%%%%%%%%%%%%%%%%%%%%%%%%%%%%%%%%%%%%%%%%
\section{INTRODUCTION}

Monocular vision has been demonstrating a flourishing field in autonomous vehicles.\\ Several applications has presented excellent 
solutions to problems presented nowadays. Therefore, this research contributes with an innovation, using Particle Image 
Velocimetry(PIV)\cite{Bastiaans} and Pearson's Correlation Coefficient(PCC)\cite{Miranda Neto}.\\ The proposal is to follow objects in scene and
stipulate its relative velocity using the both techniques cited above. These parameters are calculated in 2 and 3 dimensions, and
generated a coefficient relative to velocity of approaching and departure of objects.

Matlab is software used to simulations. PIV was the most important point of the beginning this project, considering numerous applications.\\ 
Generally, this technique is utilized to calculate the field of velocity in fluids. Thus, it's possible to calcule the field
of velocity of any objects that moves in scene. PIV was adjusted for situation to autonomous vehicles, using PCC and bank of dates of 
KITTI\cite{Geiger}.

\section{THEORETICAL FUNDAMENT}

\subsection{PEASON CORRELATION COEFFICENT - PCC}

PCC is used in different fields, like: stathistical analyses, pattern recognition and computer vision. 
Applications include disparity measurement, object recognition and comparing two images. The followed equation
describes PCC for monochrome digital images\cite{Eugene}:

\begin{center}
$$
r_i = \frac{\sum\limits_i (x_i-x_m)(y_i-y_m)}{\sqrt{\sum\limits_i (x_i-x_m)^2} \sqrt{\sum\limits_i (y_i-y_m)^2}}
$$
\end{center}

Where $x_i$ is the intensity of the i th pixel in image 1, $y_i$ is the intensity of the i th pixel in image 2, $x_m$ is the mean intensity of
image 1, and $y_m$ is the mean intensity of image 2 \cite{Miranda Neto}.

\subsection{PARTICLE IMAGE VELOCIMETRY - PIV}

PIV is a method of determining velocity fields from images of seeded flows\cite{Bastiaans}.
This technique is used to measure velocities of part or entire image. Its results is given by 
field of vector, demonstrating direction, sense and intensity of velocity in each particles. Therefore,
it is possible to calculate the velocities of any part of image with two frames, for exemple.\\
PIV is powerful technique, and current researches support the vast applications \cite{Miranda Neto, Story, Xu} 
mainly involving fluids.

\section{SYSTEM DESCRIPTION}

Diagrama1
 %A gente vai explicar o algoritmo como uma caixa fechada , que coisa entra e que coisa sai
 %e os parametros a sintonizar.
 % como usar ele quando implementado, como se fosse uma caixa preta.
 
 \section{ALGORITHM DESCRIPTION}
%DiagramaX

\subsection{MULTI-RESOLUTION MATCH CRITERIA}
%onde estava, onde esta agora
%que tamanho tinha que tamanho tem.
\subsubsection{MULTIPLAYER 3D APPROXIMATION}
%usa Multi-resolution match criteria e explica isso dos tamanhos

\subsubsection{FACTOR OF APPROACHING - RELATIVE VELOCITY}


\subsection{RENEW ROI CRITERIA}
%Diagrama2


% descri��o do sistemA
\section{NUMERICAL RESULTS}
%testes com diferentes parametros
% tabelas e graficos

\section{CONCLUSIONS}

PIV has presented satisfactory results. Different kinds of information that can be concluded, like: estimate collision, tracking of
objects in 2 or 3 dimensions and factor of approaching and removal. The simulations in Matlab has given promissories results: (TABLES
and GRAPHICS).

\addtolength{\textheight}{-12cm}   % This command serves to balance the column lengths
                                  % on the last page of the document manually. It shortens
                                  % the textheight of the last page by a suitable amount.
                                  % This command does not take effect until the next page
                                  % so it should come on the page before the last. Make
                                  % sure that you do not shorten the textheight too much.

\section*{ACKNOWLEDGMENT}

%FAPEMIG\\
%numero de bolsa\\
%numero de projeto\\
%numero de aluno


\begin{thebibliography}{99}

\bibitem{c1} G. O. Young, �Synthetic structure of industrial plastics (Book style with paper title and editor),� 	in Plastics, 2nd ed. vol. 3, J. Peters, Ed.  New York: McGraw-Hill, 1964, pp. 15�64.
\bibitem{c2} W.-K. Chen, Linear Networks and Systems (Book style).	Belmont, CA: Wadsworth, 1993, pp. 123�135.
\bibitem{c3} H. Poor, An Introduction to Signal Detection and Estimation.   New York: Springer-Verlag, 1985, ch. 4.
\bibitem{c4} B. Smith, �An approach to graphs of linear forms (Unpublished work style),� unpublished.
\bibitem{c5} E. H. Miller, �A note on reflector arrays (Periodical style�Accepted for publication),� IEEE Trans. Antennas Propagat., to be publised.
\bibitem{c6} J. Wang, �Fundamentals of erbium-doped fiber amplifiers arrays (Periodical style�Submitted for publication),� IEEE J. Quantum Electron., submitted for publication.
\bibitem{c7} C. J. Kaufman, Rocky Mountain Research Lab., Boulder, CO, private communication, May 1995.
\bibitem{c8} Y. Yorozu, M. Hirano, K. Oka, and Y. Tagawa, �Electron spectroscopy studies on magneto-optical media and plastic substrate interfaces(Translation Journals style),� IEEE Transl. J. Magn.Jpn., vol. 2, Aug. 1987, pp. 740�741 [Dig. 9th Annu. Conf. Magnetics Japan, 1982, p. 301].
\bibitem{c9} M. Young, The Techincal Writers Handbook.  Mill Valley, CA: University Science, 1989.
\bibitem{c10} J. U. Duncombe, �Infrared navigation�Part I: An assessment of feasibility (Periodical style),� IEEE Trans. Electron Devices, vol. ED-11, pp. 34�39, Jan. 1959.
\bibitem{c11} S. Chen, B. Mulgrew, and P. M. Grant, �A clustering technique for digital communications channel equalization using radial basis function networks,� IEEE Trans. Neural Networks, vol. 4, pp. 570�578, July 1993.
\bibitem{c12} R. W. Lucky, �Automatic equalization for digital communication,� Bell Syst. Tech. J., vol. 44, no. 4, pp. 547�588, Apr. 1965.
\bibitem{c13} S. P. Bingulac, �On the compatibility of adaptive controllers (Published Conference Proceedings style),� in Proc. 4th Annu. Allerton Conf. Circuits and Systems Theory, New York, 1994, pp. 8�16.
\bibitem{c14} G. R. Faulhaber, �Design of service systems with priority reservation,� in Conf. Rec. 1995 IEEE Int. Conf. Communications, pp. 3�8.
\bibitem{c15} W. D. Doyle, �Magnetization reversal in films with biaxial anisotropy,� in 1987 Proc. INTERMAG Conf., pp. 2.2-1�2.2-6.
\bibitem{c16} G. W. Juette and L. E. Zeffanella, �Radio noise currents n short sections on bundle conductors (Presented Conference Paper style),� presented at the IEEE Summer power Meeting, Dallas, TX, June 22�27, 1990, Paper 90 SM 690-0 PWRS.
\bibitem{c17} J. G. Kreifeldt, �An analysis of surface-detected EMG as an amplitude-modulated noise,� presented at the 1989 Int. Conf. Medicine and Biological Engineering, Chicago, IL.
\bibitem{c18} J. Williams, �Narrow-band analyzer (Thesis or Dissertation style),� Ph.D. dissertation, Dept. Elect. Eng., Harvard Univ., Cambridge, MA, 1993. 
\bibitem{c19} N. Kawasaki, �Parametric study of thermal and chemical nonequilibrium nozzle flow,� M.S. thesis, Dept. Electron. Eng., Osaka Univ., Osaka, Japan, 1993.
\bibitem{c20} J. P. Wilkinson, �Nonlinear resonant circuit devices (Patent style),� U.S. Patent 3 624 12, July 16, 1990. 






\end{thebibliography}




\end{document}
