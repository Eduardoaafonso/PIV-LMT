\documentclass[a4paper,10pt]{article}
\usepackage[utf8]{inputenc}

\title{}
\author{}

\date{}

\begin{document}

\maketitle

\begin{abstract}

\end{abstract}

\section{Introduction}
%%%%%%%State of art
Monocular vision has been demonstrating a flourishing field in autonomous vehicles.\\ Several applications has presented excellent 
solutions to problems presented nowadays. Therefore, this research contributes with an innovation, using Particle Image 
Velocimetry(PIV)\cite{Bastiaans} and Pearson's Correlation Coefficient(PCC)\cite{Miranda Neto}.\\ The proposal is follow objects in scene and
stipulate its relative velocity utilizing the both techniques cited above.These parameters are calculated in 2 and 3 dimensions and
generate a coefficient relative to velocity of approaching and removal of objects.

Matlab is software used to simulations. PIV went the most important point of the beginning this project, considering the applications 
with it.\\ Generally, this technique is utilised to calculate the field of velocity in fluids. Thus, it's possible to calcule the field
of velocity of any objects that moves in scene. PIV was adjusted for situation to autonomous vehicles, using PCC and bank of dates of 
KITTI\cite{Geiger}.

\section{Theoretical fundament}
\subsection{Pearson correlation}
\subsection{PIV}

\section{System description}

Diagrama1
 %A gente vai explicar o algoritmo como uma caixa fechada , que coisa entra e que coisa sai
 %e os parametros a sintonizar.
 % como usar ele quando implementado, como se fosse uma caixa preta.
\section{Algorithm description}
DiagramaX

\subsection{Multi-resolution match criteria}
onde estava, onde esta agora
que tamanho tinha que tamanho tem.
\subsubsection{Multilayer 3D approximation}
usa Multi-resolution match criteria e explica isso dos tamanhos

\subsubsection{factor of approaching - Relative velocity}


\subsection{Renew ROI criteria}
Diagrama2


% descrição do sistemA
\section{Numerical results}
%testes com diferentes parametros
% tabelas e graficos

\section{Conclusion}
PIV has presented satisfactory results. Different kinds of information that can be concluded, like: estimate collision, tracking of
objects in 2 or 3 dimensions and factor of approaching and removal. The simulations in Matlab has given promissories results: (TABLES
and GRAPHICS).

PIV is great tool to analizy tracking and velocity of objects and it shows an exepcional field to develop different ways to
solve the problems of applicability in autonomous vehicles.
\section{Acknowledgement}
FAPEMIG\\
numero de bolsa\\
numero de projeto\\
numero de aluno

\begin{thebibliography}{10}
        \bibitem{Aoude} AOUDE, Georges S. et al. \textsl{Sampling-Based Threat Assessment Algorithms for Intersection Collisions Involving 
        Errant Drivers}. IFAC Symposium on Intelligent Autonomous Vehicles, 2010.
        
	\bibitem{Bastiaans} BASTIANNS, Rob J .M. \textsl{Cross-correlation PIV; theory, implementation and accuracy}. 
        Eindhoven: Technische Universiteit Eindhoven, 2000. - EUT Report 99-W-OOl. - ISBN: 90-386-2851-X.
        
        \bibitem{Geiger} GEIGER, Andreas et al.
        \textsl{Vision meets Robotics: The KITTI Dataset}. International Journal of Robotics Research (IJRR), 2013.
        
        \bibitem{Jones} JONAS, Thomas.\textsl{Real-Time Probabilistic Collision Avoidance for Autonomous Vehicles, Using Order Reductive 
        Conflict Metrics}. Submitted to the Department of Aeronautics and Astronautics
	in partial fulfillment of the requirements for the degree of Doctor of Philosophy. Massachusetts Institute of Technology. June, 2003.
        
        \bibitem{Miranda Neto} MIRANDA NETO, Arthur et al. \textsl{Image Processing Using Pearson’s Correlation Coefficient: 
        Applications on Autonomous Robotics}. 
        Autonomous Robot Systems (Robotica), 2013 13th International Conference on, 2013.
	
	\bibitem{Woerner} WOERNER, Kyle.\textsl{COLREGS-Compliant Autonomous Collision Avoidance Using Multi-Objective Optimization
	with Interval Programming}. Submitted to the Department of Mechanical Engineering in partial fulfillment of the requirements for the 
	degrees of Naval Engineer and Master of Science in Mechanical Engineering. Massachusetts Institute of Technology. June, 2014.
	
\end{thebibliography}

\end{document}
